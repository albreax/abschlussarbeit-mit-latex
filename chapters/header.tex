% Diplomarbeit mit LaTeX
% ===========================================================================
% This is part of the book "Diplomarbeit mit LaTeX".
% Copyright (c) 2002, 2003, 2005, 2007, 2008 Tobias Erbsland
% Copyright (c) 2005, 2006 Andreas Nitsch
% See the file main.tex for copying conditions.
%

% A. DOKUMENTKLASSE
% ---------------------------------------------------------------------------

%  1. Definieren der Dokumentklasse.
%     Wir verwenden die KOMA-Script Klasse 'scrbook' für ein Buch.
%
\documentclass[%
	pdftex,%              PDFTex verwenden da wir ausschliesslich ein PDF erzeugen.
	a4paper,%             Wir verwenden A4 Papier.
	oneside,%             Einseitiger Druck.
	12pt,%                Grosse Schrift, besser geeignet für A4.
	parskip=half,%        Halbe Zeile Abstand zwischen Absätzen.
	%chapterprefix,%       Kapitel mit 'Kapitel' anschreiben.
	headsepline,%         Linie nach Kopfzeile.
	footsepline,%         Linie vor Fusszeile.
	bibliography=totoc,%  Literaturverzeichnis im Inhaltsverzeichnis nummeriert einfügen.
	index=totoc,%         Index ins Inhaltsverzeichnis einfügen.
	ngerman%              Sprache des Dokumentes
]{scrbook}
% das package listings benutzt altes komascript, siehe
% https://tex.stackexchange.com/questions/51867/koma-warning-about-toc
\usepackage{scrhack}
% tabellen mit automatischem zeilenumbruch
\usepackage{tabularx}
% avoid overfull boxes on page numbers >99, see 
% https://tex.stackexchange.com/questions/49887/overfull-hbox-warning-for-toc-entries-when-using-memoir-documentclass
% http://compgroups.net/comp.text.tex/page-number-alignment-in-koma-script-table-of-c/1914765
\makeatletter
\renewcommand*{\@tocrmarg}{2.55em}
\renewcommand*{\@pnumwidth}{2.2em}
\makeatother

%  2. Festlegen der Zeichencodierung des Dokuments und des Zeichensatzes.
%     Wir verwenden 'Latin1' (ISO-8859-1) für das Dokument,
%     und die 'T1' codierung für die Schrift.
%
\usepackage[utf8]{inputenc}
\usepackage[T1]{fontenc}

%  3. Packet für die Index-Erstellung laden.
%
\usepackage{makeidx}

%  4. Paket für die Lokalisierung ins Deutsche laden.
%     Wir verwenden neue deutsche Rechtschreibung mit 'ngerman'.
%
\usepackage[main=ngerman,french]{babel}

%  5. Paket für Anführungszeichen laden.
%     Wir setzen den Stil auf 'swiss', und verwenden so die Schweizer Anführungszeichen.
%
\usepackage[style=swiss]{csquotes}

%  6. Paket für erweiterte Tabelleneigenschaften.
%
\usepackage{array}

%  7. Paket um Grafiken im Dokument einbetten zu können.
%     Im PDF sind GIF, PNG, und PDF Grafiken möglich.
%
\usepackage{graphicx}

%  8. Pakete für mathematischen Textsatz.
%
\usepackage{amsmath}
\usepackage{amssymb}
\usepackage{dsfont}
%\usepackage{mathtools}

%  9. Paket um Textteile drehen zu können.
%
\usepackage{rotating}

% 10. Paket für Farben an verschieden Stellen. 
%     Wir definieren zusätzliche benannte Farben.
%
\usepackage{color}

% 11. Paket für spezielle PDF features.
%
\usepackage[%
	pdftitle={Diplomarbeit mit LaTeX},%                        Titel des PDF Dokuments.
	pdfauthor={Tobias Erbsland, Andreas Nitsch},%              Autor des PDF Dokuments.
	pdfsubject={Eine Einführung in LaTeX},%                    Thema des PDF Dokuments.
	pdfcreator={MiKTeX, LaTeX with hyperref and KOMA-Script},% Erzeuger des PDF Dokuments.
	pdfkeywords={Diplomarbeit, Anleitung, Einstieg,%           Schlüsselwörter für das PDF.
		LaTeX, MiKTeX, Einführung, TeXnicCenter, Windows,%     (Diese werden von Suchmaschinen
		Installation, Start, Tutorial},%                        auch für PDF Dokumente indexiert.)
	pdfpagemode=UseOutlines,%                                  Inhaltsverzeichnis anzeigen beim Öffnen
	pdfdisplaydoctitle=true,%                                  Dokumenttitel statt Dateiname anzeigen.
	pdflang=de%                                               Sprache des Dokuments.
]{hyperref}

% 12. Paket um Quellcode sauber zu formatieren.
%     Mit der option 'savemem' verschieben wir das laden von 
%     einzelnen Programmiersprachen auf einen späteren Zeitpunkt.
%
\usepackage[savemem]{listings}

% 13a. Privates Paket für die Schriftart 'Goudy Sans' laden.
%      Dieses Paket ist nur für die publizierte Version des Dokuments gedacht
%      und an dieser Stelle mit den nachfolgenden Anweisungen auskommentiert.
%
%\usepackage{goudysans}

%
% 13a. Font 'Latin Modern Family' verwenden.
%      Verwende dieses Paket wenn du DML selbst kompilierst.
%
\usepackage{lmodern}

% B. EINSTELLUNGEN
% ---------------------------------------------------------------------------

%  1. Definieren von eigenen benannten Farben.
%     Für spätere Verwendung in dem Dokument, definieren wir einzelne
%     benannte Farben.
%
\definecolor{LinkColor}{rgb}{0,0,0.5}
\definecolor{ListingBackground}{rgb}{0.85,0.85,0.85}

%  2. KOMA-Script Option, Zeilenumbruch bei Bildbeschreibungen.
%
\setcapindent{1em}

%  3. Stil der Kopf- und Fusszeilen.
%     Wir aktivieren mit 'headings' laufende Seitentitel.
%
\pagestyle{headings}

%  4. Stil der Überschriften auf normale Schrift.
%     Wir verwenden für die Überschriften den selben Font wie für den Text.
%
\setkomafont{sectioning}{\normalfont\bfseries}       % Titel mit Normalschrift
\setkomafont{captionlabel}{\normalfont\bfseries}     % Fette Beschriftungen 
\setkomafont{pagehead}{\normalfont\itshape}          % Kursive Seitentitel
\setkomafont{descriptionlabel}{\normalfont\bfseries} % Fette Beschreibungstitel

%
%  5. Farbeinstellungen für die Links im PDF Dokument.
%
\hypersetup{%
	colorlinks=true,%        Aktivieren von farbigen Links im Dokument (keine Rahmen)
	linkcolor=LinkColor,%    Farbe festlegen.
	citecolor=LinkColor,%    Farbe festlegen.
	filecolor=LinkColor,%    Farbe festlegen.
	menucolor=LinkColor,%    Farbe festlegen.
	urlcolor=LinkColor,%     Farbe von URL's im Dokument.
	bookmarksnumbered=true%  Überschriftsnummerierung im PDF Inhalt anzeigen.
}

%  6. Einstellungen für das 'listings' Paket.
%
\lstloadlanguages{TeX} % TeX sprache laden, notwendig wegen option 'savemem'
\lstset{%
	language=[LaTeX]TeX,     % Sprache des Quellcodes ist TeX
	literate=                % https://en.wikibooks.org/wiki/LaTeX/Source_Code_Listings#Encoding_issue
		{á}{{\'a}}1 {é}{{\'e}}1 {í}{{\'i}}1 {ó}{{\'o}}1 {ú}{{\'u}}1
		{Á}{{\'A}}1 {É}{{\'E}}1 {Í}{{\'I}}1 {Ó}{{\'O}}1 {Ú}{{\'U}}1
		{à}{{\`a}}1 {è}{{\`e}}1 {ì}{{\`i}}1 {ò}{{\`o}}1 {ù}{{\`u}}1
		{À}{{\`A}}1 {È}{{\'E}}1 {Ì}{{\`I}}1 {Ò}{{\`O}}1 {Ù}{{\`U}}1
		{ä}{{\"a}}1 {ë}{{\"e}}1 {ï}{{\"i}}1 {ö}{{\"o}}1 {ü}{{\"u}}1
		{Ä}{{\"A}}1 {Ë}{{\"E}}1 {Ï}{{\"I}}1 {Ö}{{\"O}}1 {Ü}{{\"U}}1
		{â}{{\^a}}1 {ê}{{\^e}}1 {î}{{\^i}}1 {ô}{{\^o}}1 {û}{{\^u}}1
		{Â}{{\^A}}1 {Ê}{{\^E}}1 {Î}{{\^I}}1 {Ô}{{\^O}}1 {Û}{{\^U}}1
		{Ã}{{\~A}}1 {ã}{{\~a}}1 {Õ}{{\~O}}1 {õ}{{\~o}}1
		{œ}{{\oe}}1 {Œ}{{\OE}}1 {æ}{{\ae}}1 {Æ}{{\AE}}1 {ß}{{\ss}}1
		{ű}{{\H{u}}}1 {Ű}{{\H{U}}}1 {ő}{{\H{o}}}1 {Ő}{{\H{O}}}1
		{ç}{{\c c}}1 {Ç}{{\c C}}1 {ø}{{\o}}1 {å}{{\r a}}1 {Å}{{\r A}}1
		{€}{{\euro}}1 {£}{{\pounds}}1 {«}{{\guillemotleft}}1
		{»}{{\guillemotright}}1 {ñ}{{\~n}}1 {Ñ}{{\~N}}1 {¿}{{?`}}1,
	numbers=left,            % Zelennummern links
	stepnumber=1,            % Jede Zeile nummerieren.
	numbersep=5pt,           % 5pt Abstand zum Quellcode
	numberstyle=\tiny,       % Zeichengrösse 'tiny' für die Nummern.
	breaklines=true,         % Zeilen umbrechen wenn notwendig.
	breakautoindent=true,    % Nach dem Zeilenumbruch Zeile einrücken.
	postbreak=\space,        % Bei Leerzeichen umbrechen.
	tabsize=2,               % Tabulatorgrösse 2
	basicstyle=\ttfamily\footnotesize, % Nichtproportionale Schrift, klein für den Quellcode
	showspaces=false,        % Leerzeichen nicht anzeigen.
	showstringspaces=false,  % Leerzeichen auch in Strings ('') nicht anzeigen.
	extendedchars=true,      % Alle Zeichen vom Latin1 Zeichensatz anzeigen.
	backgroundcolor=\color{ListingBackground}} % Hintergrundfarbe des Quellcodes setzen.

% C. NEUE MAKROS UND UMGEBUNGEN
% ---------------------------------------------------------------------------

%  1. Umgebung für Änerungsliste mit einem speziellen Aufzählungszeichen.
%
\newenvironment{ListChanges}%
	{\begin{list}{$\diamondsuit$}{}}%
	{\end{list}}

%  2. Ersatz für die \LaTeX und \TeX Befehle für korrekte Darstellung.
%     Wir verwenden die 'Latin Modern Family' ('lm') als Font, da diese im
%     vergleich zu 'Computer Modern' ('cm') auch PostScript Dateien
%     anbieten, was zu einer schöneren Darstellung im PDF führt.
%
\newcommand{\DMLLaTeX}{{\fontfamily{lmr}\selectfont\LaTeX}}
\newcommand{\DMLTeX}{{\fontfamily{lmr}\selectfont\TeX}}

\def\AmS{$\mathcal{A}$\kern-.1667em\lower.5ex\hbox
    {$\mathcal{M}$}\kern-.125em$\mathcal{S}$}
\def\AmSmath{\AmS{}math}

\newcommand{\urlindent}[2][]{%
	\begin{addmargin}[1em]{0em}%
		\url{#2}{#1}%
	\end{addmargin}%
}

% D. AKTIONEN
% ---------------------------------------------------------------------------

%  1. Index erzeugen.
%
\makeindex

% E. SILBENTRENNUNG
% ---------------------------------------------------------------------------

\hyphenation{De-zi-mal-trenn-zeichen In-stal-la-ti-ons-as-sis-tent}

