
% Diplomarbeit mit LaTeX
% ===========================================================================
% This is part of the book "Diplomarbeit mit LaTeX".
% Copyright (c) 2002-2005, 2007 Tobias Erbsland
% Copyright (c) 2005 Andreas Nitsch
% See the file diplomarbeit_mit_latex.tex for copying conditions.
%

\chapter{Literaturempfehlungen}
\index{Literatur}\index{Bucher@Bücher}

\section{Bücher und Internetadressen}

Setzt du konsequent die Hilfsmittel ein, die \DMLLaTeX \ dir bietet, so wirst du auf jeden Fall ein optisch ansprechendes Dokument erhalten. Wenn du dabei auch noch die Regeln beachtest und dich um Warnungen und Fehlermeldungen kümmerst, dann erhälst du ein optisch überzeugendes Dokument.

Ob deine Diplomarbeit für den Leser besonders spannend ist, hängt sowohl vom jeweiligen Thema und von den Interessen des Lesers ab. Trotzdem kannst du noch einiges optimieren um deine Arbeit für den Leser\footnote{Besonders für diejenigen, die dir eine Note dafür geben werden} spannender und sicher auch unterhaltsamer zu gestalten.

Mit \enquote{einfachen Wörtern} und \enquote{schönen Sätzen} kann der Leser dein Werk in vollen Zügen genießen.

Falls aber Schreiben nicht zu deinem Stärken zählt, könnten dir die folgenden Literaturtipps eine gute Hilfe sein. Einige davon kannst du kostenlos über das Internet beziehen.

\begin{itemize}
	\item{Claudia Fritsch: ''Schreiben für die Leser''\cite{fritsch:schreiben_fuer_die_Leser}}
	\item{Wolf Schneider: ''Deutsch fürs Leben''\cite{schneider:deutsch_fuers_leben}}
\end{itemize}

Weitere und ausführlichere \footnote{...dafür aber nicht so speziell auf das Erstellen einer Diplomarbeit ausgelegte} Literatur zu \DMLLaTeX \ findest du  kostenlos im Internet:

\begin{itemize}
	\item{Manuela Jürgens: \enquote{\DMLLaTeX - eine Einführung und ein bisschen mehr}\cite{juergens:latex1}}
	\item{Manuela Jürgens: \enquote{\DMLLaTeX - fortgeschrittene Anwendungen}\cite{juergens:latex2}}
\end{itemize}

Zudem gibt es viele gute Bücher welche weitere Details von \DMLLaTeX \ beschreiben. Folgende Bücher können als Standardwerke in Sachen \LaTeX \ bezeichnet werden:

\begin{itemize}
	\item{Frank Mittelback, Michael Goossens: \enquote{Der \DMLLaTeX \ Begleiter}\cite{DerLaTeXBegleiter}}
	\item{Helmut Kopka: \enquote{\DMLLaTeX: Band 1 - Eine Einführung}\cite{kopka:band1}}
	\item{Helmut Kopka: \enquote{\DMLLaTeX: Band 2 - Ergänzungen}\cite{kopka:band2}}
	\item{Helmut Kopka: \enquote{\DMLLaTeX: Band 3 - Erweiterungen}\cite{kopka:band3}}
	\item{Leslie Lamport: \enquote{\DMLLaTeX, A Document Preparation System, User's Guide and Reference Manual}\cite{lamport:latex}}
\end{itemize}




