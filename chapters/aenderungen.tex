%
% Diplomarbeit mit LaTeX
% ===========================================================================
% This is part of the book "Diplomarbeit mit LaTeX".
% Copyright (c) 2002-2008 Tobias Erbsland, Andreas Nitsch
% See the file diplomarbeit_mit_latex.tex for copying conditions.
%

\chapter{Änderungen an diesem Dokument}
\label{sec:aendeungen}

\begin{labeling}{(MMM)}
	\item[(te)] Tobias Erbsland (inklusive aller Einträge ohne Namen)
	\item[(an)] Andreas Nitsch
	\item[(se)] seth
\end{labeling}


\begin{labeling}{Version 99.99 / 99. September 9999} % Spez. komascript Umgebung
	\item[Version 1.12.1 / 25. Juli 2020]
		\begin{ListChanges}
		\item (se) Einige veraltete LaTeX-Konstrukte modernisiert, Links aktualisiert, over-/underfull Boxes aufgelöst, Eingangskapitel aktualisiert, kleinere Korrekturen 
		\end{ListChanges}
	\item[Version 1.12 / 20. März 2008]
		\begin{ListChanges}
			\item (te) Im Text \enquote{File} durch \enquote{Datei} ersetzt.
			\item (te) Einen Hinweis zu fett gesetzten Texten ergänzt.
			\item (te) Installation überprüft und Link zu MiKTeX installer angepasst.
			\item (te) Kapitel \enquote{Literaturempfehlungen} überarbeitet.
				Sprache vereinfacht, Rechtschreibprüfungsabschnitt ins Konfigurationskapitel verschoben. 
				Buch \enquote{Der \DMLLaTeX"=Begleiter} hinzugefügt.
			\item (te) Kapitel \enquote{Mathematischer Textsatz} (vorher \enquote{Formeln}) überarbeitet.
			\item (te) \enquote{header.tex} überarbeitet. Detailiertere Kommentare eingefügt und Optionen der Pakete angepasst oder verändert damit DML mit MiKTeX 2.6 optimal kompiliert werden kann.
			\item (te) Statt einem richtigen @ Zeichen, \enquote{at} in die E-Mail Adressen im Titel engefügt. Da Spammer offensichtlich auch PDFs durchsuchen.
			\item (te) Neue Umgebung \verb|\DMLLaTeX| und \verb|\DMLTeX| erstellt, welche unabhängig von der gewählten Schriftart das \DMLLaTeX- und das \DMLTeX"=Logo korrekt setzen.
		\end{ListChanges}
	\item[Version 1.11 / 2. April 2006]
		\begin{ListChanges}
			\item (te) Kapitel über das Paket \enquote{csquotes} von Uwe Bieling in das neue Kapitel \enquote{Nützliche Pakete} eingefügt.
			\item (te) Neue Version da wieder viele Ändeungen an dem Dokument gemacht wurden. Leider haben die entsprechenden Personen diese hier nicht eingetragen.
		\end{ListChanges}
	\item[Version 1.10 / 20. Juli 2005]
		\begin{ListChanges}
			\item (te) Vorwort um Anfrage nach Hilfe ergänzt.
			\item (an) Kapitel über Literaturverzeichnisse und Glossare hinzugefügt.
			\item (an) Kapitel mit Literaturempfehlungen und formalen Hilfsmitteln hinzugefügt.
			\item (an) Beispieldokument einer Diplomarbeit zu den Listings hinzugefügt.
			\item (an) Diverse Rechtschreibfehler berichtigt.
		\end{ListChanges}
	\item[Version 1.9 / 20. Oktober 2003]
		\begin{ListChanges}
			\item Kapitel \enquote{Deutsche Anführungszeichen einstellen} hinzugefügt (nach einer Mail von Christian Günther).
			\item Erklärung zur Unterscheidung von schweizerischen und französischen Anführungszeichen im Abschnitt~\ref{sec:anfuehrungszeichen} hinzugefügt\\
			(nach einem Hinweis von Erich Hohermuth).
			\item Im Abschnitt \ref{sec:erstesbeispiel} habe ich mehr zu der \enquote{T1} Codierung hinzugefügt (nach einer Mail von Christian Günther).
			\item Ich habe überall die Zeile\\ 
				\texttt{\textbackslash usepackage[german,ngerman]\{babel\}}\\
				durch \texttt{\textbackslash usepackage\{ngerman\}} ersetzt (nach einer Anregung von Christian Günther).
			\item In der Hauptdatei dieses Dokuments habe ich die \texttt{\textbackslash input} durch \texttt{\textbackslash include} Befehle ersetzt.
			\item Im Kapitel \ref{sec:aufbaugrosserdokumente} habe ich in einem Abschnitt die Unterschiede zwischen \texttt{\textbackslash input} und \texttt{\textbackslash include} erklärt.
		\end{ListChanges}
	\item[Version 1.8 / 9. Februar 2003]
		\begin{ListChanges}
			\item Rechtschreibkorrekturen und Änderungen an der Formulierung im Kapitel~\ref{sec:installation}
				(nach Anregungen von Christian Faulhammer)
			\item Satzspiegel und die Schriftgröße verändert -- versucht, Vorgaben des \KOMAScript{}"=Pakets umzusetzen. 
				Schriftgröße auf 12 Punkte erhöht. Ich hoffe, dass dadurch das Dokument am Bildschirm besser lesbar ist.
			\item Backslash korrigiert. Statt \texttt{\$\textbackslash backslash\$} verwende ich nun überall \texttt{\textbackslash textbackslash}.
			\item Den Link zu der \KOMAScript{} Webseite korrigiert.
			\item Den Ausdruck KOMA-Script an allen Stellen im Dokument durch \KOMAScript{} ersetzt.
			\item Jedes \enquote{z.B} durch \enquote{z.\,B.} ersetzt.
			\item Die \enquote{center} Umgebung bei Tabellen\\ und Grafiken durch \texttt{\textbackslash centering} ersetzt.
			\item Hinweis auf das Dokument \enquote{tabsatz} im Abschnitt \ref{sec:linienintabellen} eingefügt.
			\item Hinweis zu WYSIWYG"=Programmen im Kapitel \ref{sec:motivation}
			\item Kapitel \enquote{Dokumentklassen} um zwei Kapitel vorgezogen.
			\item Alle Dokumentklassen direkt durch \KOMAScript{}"=Dokumentklassen ersetzt. 
				Da es sich um eine deutsche Dokumentation handelt, führe ich direkt die \KOMAScript{}"=Dokumentklassen ein. 
				Diese sind für Anfänger wesentlich besser geeignet und bieten mehr Optionen.
			\item Den Hinweis für einen europäischen Absatz geändert, und die Klassenoptionen\\
			 von \KOMAScript{} eingefügt.
			\item Das Paket \enquote{pslatex} durch \enquote{mathptmx},\\
					  \enquote{helvet} und \enquote{courier} ersetzt.
			\item In Kapitel \ref{sec:separatetitelseite} den Hack für eine separate Titelseite durch einen Hinweis auf die entsprechende Klassenoption ersetzt.
			\item \texttt{\textbackslash title} und \texttt{\textbackslash author} aus diesem Dokument und aus dem Beispiellisting mit eigener Titelseite entfernt.
		\end{ListChanges}
	\item[Version 1.7 / 24. Januar 2003]
		\begin{ListChanges}
			\item Das Kapitel \enquote{Aufbau großer Dokumente} fertiggestellt.
		\end{ListChanges}
	\item[Version 1.6 / 17. Januar 2003]
		\begin{ListChanges}
			\item Das Kapitel \enquote{Dokument~-klassen}\\
			      fertiggestellt.
		\end{ListChanges}
	\item[Version 1.5 / 18. Dezember 2002]
		\begin{ListChanges}
			\item Paket \enquote{hyperref} für eine einfache Navigation innerhalb des PDF-Dokuments (Bookmarks, anklickbare Links) hinzugefügt.			
		\end{ListChanges}
	\item[Version 1.4]  
		\begin{ListChanges}
			\item Index hinzugefügt.
			\item Das Vorwort hinzugefügt.
		\end{ListChanges}
	\item[Version 1.2]
		\begin{ListChanges}
			\item Diverse Änderungen am Layout.
			\item Kapitel \enquote{Tabellen und Bilder} und \enquote{Dokumentteile} fertiggestellt.
		\end{ListChanges}
	\item[Version 1.0]
		\begin{ListChanges}
			\item Erste Vorschauversion.
		\end{ListChanges}
\end{labeling}
