
% Abschlussarbeit mit LaTeX
% ===========================================================================
% This is part of the book "Abschlussarbeit mit LaTeX".
% Copyright (c) 2002-2005 Tobias Erbsland, Andreas Nitsch
% See the file abschlussarbeit_mit_latex.tex for copying conditions.
%

\chapter{Literaturverzeichnisse und Glossare}
\label{sec:literaturverzeichnisse}

In \DMLLaTeX{} bieten sich zwei unterschiedliche Wege an, Literaturverzeichnisse
anzulegen und zu gestalten. Der einfachere Weg, der allerdings nicht sonderlich
viele Gestaltungsmöglichkeiten offen hält, ist die Nutzung der Umgebung \texttt{thebibliography}. Sollen umfangreichere Literaturverzeichnisse oder gar eine Verbindung zu einer Literatur-Datenbank genutzt werden, so
bietet sich das mit \DMLLaTeX{} mitgelieferte Bib\DMLTeX{} an, zu dessen Nutzung allerdings etwas Einarbeitung notwendig ist.

\section{Einfache Literaturverzeichnisse}
\index{Literaturverzeichnisse!Literaturverzeichnisse@einfache}

Zum Verweisen auf einen Eintrag im Literaturverzeichnis wird sowohl bei der folgenden
einfachen Verzeichniserstellung als auch bei der Erstellung eines Literaturverzeichnisses
mit Bib\DMLTeX{} der Befehl \texttt{$\backslash$cite\{marke\}} benutzt. Die Variable
\texttt{marke} bezeichnet den Namen des Werkes innerhalb des Literaturverzeichnisses.

Das eigentliche Literaturverzeichnis wird durch Einfügen der folgenden Umgebung an der
Stelle des Textes, an der es erscheinen soll, eingefügt:

\begin{verbatim}
      \begin{thebibliography}{defaultmarken}
            \bibitem{marke} Angaben zur Literaturquelle
                .
                .
                .
      \end{theblibliography}
\end{verbatim}

Die angegebene \texttt{defaultmarke} legt die Formatierung der Nummerierung fest.
\newpage

Folgendes Beispiel soll den Einsatz der Umgebung verdeutlichen:

\begin{verbatim}
Was er mit den erbeuteten Knochen zu tun hatte, wusste der
kleine Hund aus einem Buch, dass er in der Stadtbücherei
gelesen hatte: \emph{"Mein Kochen und ich"} \cite{bjarne}.
Dieses stand dort direkt neben seinem Lieblingsbuch
\emph{"Von Hund zu Hund"} \cite{katz} im Bücherregal
der Hundeliteratur.\\

\begin{thebibliography}{999}
   \bibitem{bjarne} Bjarne Friedjof Blue: "Mein Knochen
   									und ich - wie ich meinen Schatz vergrub"
   \bibitem{katz} Richard Katz: "Von Hund zu Hund"
\end{thebibliography}

\end{verbatim}

Die Ausgabe sieht folgendermaßen aus:

\fbox{\parbox[b]{14.5cm}{
	
	Was er mit den erbeuteten Knochen zu tun hatte, wusste der kleine Hund
	aus einem Buch, dass er in der Stadtbücherei gelesen hatte:
	\emph{''Mein Kochen und ich''}[1]. Dieses stand dort direkt neben
	seinem Lieblingsbuch \emph{''Von Hund zu Hund''}[2] im Bücherregal
	der Hundeliteratur.
	\vspace{0.5cm}
	\\	
	\Large{Literatur}
	\vspace{1em}
	\\
	\normalsize
	$[$1$]$ Bjarne Friedjof Blue: ''Mein Knochen und ich - wie ich meinen Schatz vergrub''\\
	$[$2$]$ Richard Katz: ''Von Hund zu Hund''
	}
}

\section{Aufwendigere Literaturverzeichnisse}
\index{Literaturverzeichnisse!Literaturverzeichnisse@aufwendigere}
In sämtlichen wissenschaftlichen Werken, zu denen die meisten Abschlussarbeiten zählen, sollte großer Wert auf ein vollständiges und den Normen genügendes Literaturverzeichnis gelegt werden. Bib\TeX \ stellt ein äußerst leistungsfähiges Tool dar, mit dessen Hilfe man automatisch Literaturverzeichnisse erstellen kann.

\subsection{Erstellen der Referenzangaben}
Die eigentlichen Informationen zur verwendeten Literatur werden in einer oder mehreren
separaten Datei(en) mit der Endung \texttt{.bib} abgelegt. Diese Dateien beinhalten für
jedes anzugebene Werk einen Eintrag, der je nach Referenzart entsprechende Attribute
besitzt.

Folgend ein Beispiel für den Eintrag eines Buches:

\begin{verbatim}
@book{bjarne:knochen,
       author={Bjarne Friedjof Blue},
       title={Mein Knochen und ich - wie ich meinen Schatz
       				vergrub},
       edition={2},
       publisher={Wuff-Verlag},
       isbn={3-12345-6789},
       month={November},
       year={2004},
       language={husky-hündisch},
}
\end{verbatim}

Auf dieses Werk würde im Text mit der Referenz \verb|\cite{bjarne:knochen}| referenziert
werden.

Als Referenzarten stehen unter anderem folgende Typen zur Verfügung:

\begin{table}[h]
	\centering
		\begin{tabular}[t]{|l|l|}
		\hline
		@book &Ein von einem Verlag publiziertes Buch\\
		\hline
		@booklet &Gedruckte Arbeit ohne einen Verleger oder eine\\
		 				&publizierende Einrichtung\\
		\hline
		@article &Ein in einem Magazin oder Journal veröffentlichter Artikel\\
		\hline
		@inbook &Teil eines Buches, ein Kapitel oder ein bestimmter Bereich\\
						&(Seiten von - bis)\\
		\hline
		@manual &Eine technische Dokumentation\\
		\hline
		@masterthesis &Abschlussarbeit\\
		\hline
		@misc &Ein Werk, das in keine andere Kategorie passt\\
		\hline
		\end{tabular}
	\caption{BiB\TeX \ Referenzarten}
	\label{tab:BiBTeXReferenzarten}
\end{table}

\newpage

Je nach Referenzart sind manche Angaben zu einem Werk zwingend erforderlich,
optional oder nicht erforderlich (bei Fehlen gibt \LaTeX \ eine Warnung aus).
Einen Überblick über die wichtigsten Attributfelder gibt folgende Tabelle:

\begin{table}[ht]
	\centering
		\begin{tabular}{|l|l|}
			\hline
			author &Name des Autors oder der Autoren\\
			\hline
			booktitle &Titel eines Buches oder eines Buchteils.Zum Verweis auf ein\\
								&ganzes Buch steht das Feld \texttt{title} zur Verfügung.\\
			\hline
			chapter &Eine Kapitelnummer oder Kapitelbezeichnung.\\
			\hline
			edition &Auf\/lage des Buches: Zahl oder ausgeschriebene Zahl.\\
			\hline
			institution &Institution, an der das Werk entstand.\\
			\hline
			journal &Name des Journals oder Magazins.\\
			\hline
			month &Monat der Veröffentlichung\\
			\hline
			pages &Eine oder mehrere Seitenzahlen oder -bereiche,\\
						&z.\,B. 42--50 oder 12, 43, 67.\\
			\hline
			publisher &Name des Verlegers.\\
			\hline
			title &Titel der Arbeit.\\
			\hline
			year &Erscheinungsjahr\\
			\hline
			ISBN &International Standard Book Number\\
			\hline
			language &Sprache, in der das Werk verfasst ist.\\
			\hline
			URL &Universal Ressource Locator, Angabe einer Adresse im Internet\\			
			\hline
		\end{tabular}
	\caption{Literatur-Attributfelder}
	\label{tab:LiteraturAttributfelder}
\end{table}

\subsection{Festlegung des Anzeigestils}
Nachdem die Literaturreferenzen angelegt und in einer oder mehreren Dateien
definiert wurden, können sie nun unter Angabe eines Anzeigestils, welcher das
genaue Aussehen des Literaturverzeichnisses definiert, in das Hauptdokument eingebunden werden:

\begin{verbatim}
...
\bibliographystyle{geralpha}
\bibliography{name_der_bib_datei_ohne_Endung}
...
\end{verbatim}

\newpage

Eine Übersicht der gebräuchlichsten Styles für deutschsprachige Literaturverzeichnisse
(Präfix \textit{ger} steht für German) soll folgende Tabelle geben:

\begin{table}[ht]
	\centering
		\begin{tabular}{|l|l|}
			\hline
				gerabbrv &$[$5$]$ Schneider, W.: \emph{Deutsch fürs Leben - was die Schule zu lehren 				 vergaß.}\\
				&\ \ \ \ Rowohlt Taschenbuch Verlag GmbH, Februar 2002.\\
			\hline
				geralpha &$[$Sch02$]$ Schneider, Wolf: \emph{Deutsch fürs Leben - was die Schule zu
				lehren}\\
				&\ \ \ \ \emph{vergaß.} Rowohlt Taschenbuch Verlag GmbH, Februar 2002.\\
			\hline
			  gerapali &$[$Schneider 2002$]$ SCHNEIDER, WOLF (2002). \emph{Deutsch fürs Leben - 					was}\\
			  &\ \ \ \ \ \emph{die Schule zu lehren vergaß.} Rowohlt Taschenbuch Verlag GmbH.\\
			\hline
				gerplain &$[$1$]$ HELMUT SCHAEBEN, MARCUS APEL: \emph{GIS 2D, 3D, 4D, nD.}\\
				&\ \ \ \ \ Informatik Spektrum, Juni 2003.\\
				&\ \ \ \ \ (im Gegensatz zu \texttt{gerabbrv} wird hier nicht nur der erstgenannte\\
				&\ \ \ \ \  Autor aufgeführt.\\
			\hline
			  gerunsrt &wie \texttt{gerabbrv}, allerdings werden die Werke nicht alphabetisch nach\\
			  &\ \ \ \ \  Autor sortiert, sondern wie in der bib-Datei aufgeführt aufgelistet.\\
			  \hline
		\end{tabular}
	\caption{Style Übersicht}
	\label{tab:style_uebersicht}
\end{table}

Um diese Styles nutzen zu können, muß das Paket \textit{germbib} installiert sein (bei MiKTeX-Standardinstallation der Fall) und \textit{bibgerm.sty} mit \textbackslash usepackage\{bibgerm\} im Dokumentenkopf eingebunden werden.

Wer mit den gegebenen Standard-Styles nicht zufrieden ist, der kann sich auch seinen
eigenen Bib\TeX-Style erstellen. Dieses zu schildern würde hier allerdings den Rahmen
sprengen, deshalb sei hier nur auf das Tutorium von Bernd Raichle \cite{raichle:bibtex_programmierung} verwiesen.

\subsection{Einbinden der Referenzen in den Text und Erstellung des
Literaturverzeichnisses}
Nachdem die eigentlichen Angaben zur verwendeten Literatur in der .bib-Datei
angelegt worden sind, müssen diese Angaben noch mit den passenden Stellen im
Text, an denen das Werk zitiert wird, verknüpft werden.
Hierzu wird der Befehl \texttt{cite{}}(für citation) verwendet:
\begin{verbatim}
	Nähere Informationen finden sich im Buch "Mein Knochen und ich -
	wie ich meinen Schatz vergrub" \cite{bjarne:knochen}
\end{verbatim}
\newpage
Die Ausgabe sieht, je nach Einstellung des Anzeigestiles (s. hierzu Tabelle
\ref{tab:style_uebersicht}) etwa wie folgt aus:
\begin{verbatim}
	Nähere Informationen finden sich im Buch "Mein Knochen und ich -
	wie ich meinen Schatz vergrub" [Bjar05].
\end{verbatim}
Um das eigentliche Literaturverzeichnis zu erstellen und einzubinden, ist folgendes Vorgehen
nötig:

Zuerst lässt man einen \DMLLaTeX"=Lauf über das gesamte Dokument laufen, wodurch
für jede Datei der Projektes eine zugehörige Datei mit der Endung \texttt{.aux}
erstellt wird. In diesen Dateien sind unter anderem die Literatureinträge
verzeichnet, auf die in dem jeweiligen Dokument verwiesen wird.
Anschließend wird das Programm \texttt{Bib\DMLTeX} aufgerufen, welches diese Einträge
sammelt, sortiert und in eine Datei mit der Endung \texttt{.bbl} schreibt.
Bib\DMLTeX{} kann im \DMLTeX nicCenter sehr bequem über den Menüpunkt \texttt{Ausgabe
$\rightarrow$ Bib\DMLTeX} aufgerufen werden. Nach einem weiteren \DMLLaTeX"=Lauf wird
das fertig sortierte und formatierte Literaturverzeichnisse in das Dokument
eingebunden.

Falls dir das dauernde Aufrufen von \DMLLaTeX{} bzw. Pdf\DMLTeX{} und die Aufrufe von
Bib\DMLTeX{} und makeindex (s. hierzu Kapitel \ref{sec:glossar}) zuviel wird, kannst du diese
Arbeitsschritte auch über eine Befehlsdatei, ein sogenanntes \texttt{Batchscript}
automatisieren. Ein Beispiel für eine solche Batchdatei findet sich in Listing
\ref{subsec:batchdatei}.
