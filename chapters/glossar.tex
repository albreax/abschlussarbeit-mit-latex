%
% Abschlussarbeit mit LaTeX
% ===========================================================================
% This is part of the book "Abschlussarbeit mit LaTeX".
% Copyright (c) 2002, 2003, 2005 Tobias Erbsland, Andreas Nitsch
% See the file main.tex for copying conditions.
%

\section{Glossare}
\label{sec:glossar}

Bei größeren Dokumenten, insbesondere zu komplexen Themen, kann es recht
sinnvoll sein, Begriffe, die nicht jedem geläufig sind und oft benutzt
werden, gesondert zu erklären.

Selbst beim Schreiben einer Abschlussarbeit kann es für den Autor selbst
sehr sinnvoll sein, Begrifflichkeiten zu erklären, um sich selber vollständig
über ihre Bedeutung klar zu werden. Dem Leser jedenfalls wird mit einem Glossar
unter Umständen eine große (Verständnis-) Hilfe geboten und lästiges
Durchsuchen der gesamten Arbeit nach einer Begriffserklärung wird vermieden.
\vspace{1em}
\\
Zum Erstellen eines Glossars gehören, wie beim Erstellen des Literaturverzeichnisses,
 prinzipiell drei Schritte.
Im ersten Schritt werden die einzelnen Glossareinträge innerhalb des Textes
angelegt. Durch einen \DMLLaTeX"=Lauf werden diese Glossareinträge in
einer Datei mit der Endung ''.glo'' gesammelt, durch den Aufruf des
Zusatzprogrammes \emph{makeindex} sortiert und in das \DMLLaTeX"=Dokument
eingebunden. Durch einen weiteren \DMLLaTeX"=Lauf wird hiernach das fertige
Dokument erzeugt.
\vspace{1em}
\\
Die Glossareinträge können an jeder beliebigen Stelle eines Dokumentes erstellt werden.
Die Definition erfolgt durch den Befehl \texttt{glossary}:
\begin{verbatim}
	\glossary{
						name={Knochen},
						description={Lieblingsspeise eines jeden Hundes. Besonderer
												 Beliebtheit erfreuen sich Rinderknochen.}
	}
\end{verbatim}
Nach einem  \DMLLaTeX"=Durchlauf werden die Zwischendateien mit den Endungen ''.glo'' und
''.ist''erzeugt, welche die aus den Dateien extrahierten Einträgen bzw. die
Formatierungsbeschreibung des Glossars enthalten. Um diese Formatierung auf die extrahierten
Einträge anzuwenden, wird das Programm \texttt{makeindex} benutzt, dessen Aufruf leider
etwas langatmig ist. In der Windows-Eingabeaufforderung (Start $\rightarrow$ Alle Programme
$\rightarrow$ Zubehör $\rightarrow$ Eingabeaufforderung) ist folgende Kommandozeile einzugeben:
\begin{verbatim}
		makeindex -s abschlussarbeit_akki.ist -t abschlussarbeit_akki.glg -o
								 abschlussarbeit_akki.gls abschlussarbeit_akki.glo
\end{verbatim}
Makeindex erzeugt eine Datei mit der Endung ''.gls'', in welcher das fertig formatierte Glossar
enthalten ist. Dieses kann mit dem Befehl
\begin{verbatim}
	\renewcommand{\glossaryname}{Glossar}
	\printglossary
\end{verbatim}
an jeder beliebigen Stelle der Abschlussarbeit eingefügt werden und wird dort nach einem weiteren
\DMLLaTeX"=Lauf ausgegeben. Das Kommando \texttt{renewcommand} dient an dieser Stelle dazu, die
automatisch eingefügte Überschrift des Glossars neu zu setzen, da diese standardmäßig auf
englich erscheint (''Glossary'') und bei einer deutschsprachigen Abschlussarbeit besser in deutsch
gehalten sein sollte.

Ein Beispiel für das Einbinden eines Glossars findet sich in
\cref{subsec:listing_hauptdokument}, ein Beispiel für das Anlegen von Glossareinträgen in 
\cref{subsec:zweites_kapitel} und zum automatisierten Ablauf des Übersetzungsvorganges wird auf
die Batchdatei in \cref{subsec:batchdatei} verwiesen.

\subsection{Formatierungsmöglichkeiten des Glossars}
Neben der Änderung der Glossarüberschrift gibt es noch weitere Einstellungen, mit denen man das
Aussehen des Glossars beeinflussen kann. Folgende Einstellungen können im Header vorgenommen
(s. \cref{subsec:header}) werden und bestimmen, wie das glossary-Paket angewendet wird:

\begin{tabularx}{\textwidth}{lX}
			style &Der Glossar-Stil. Mögliche Werte sind:\\
						&\textbf{list}: stellt Begriffe (fettgedruckt) und Erklärung in einer Zeile dar.\\
						&\textbf{altlist}: stellt den Begriff fettgedruckt dar, Erklärung folgt versetzt auf der nächsten Zeile\\
						&\textbf{super}: nutzt die \texttt{supertabular}- Umgebung für das Glossar.\\
						&\textbf{long}: nutzt die \texttt{longtable}- Umgebung für das Glossar (default).\\
			header &Setzt die Überschriften der Begriffs- und Erklärungsspalten. Mögliche Werte sind:\\
					  &\textbf{none}: Spalten haben keine Überschriften (default).\\
					  &\textbf{plain}: Spalten haben Überschriften.\\
		  border &Rahmen um das Glossar. Mögliche Werte sind:\\
		  			&\textbf{none}: Glossar ist nicht eingerahmt (default).\\
		  			&\textbf{plain}: Körper des Glossars wird eingerahmt.\\		  
			cols	& Anzahl der Spalten des Glossars. Mögliche Werte sind:\\ 
						&\textbf{2}: Die Begriffsbezeichnung ist in der ersten, Erklärung und zugehörige Seitenzahl in der zweiten Glossarspalte (default).\\ 
						&\textbf{3}: Die Begriffsbezeichnung ist in der ersten, die Erklärung in der zweiten und die zugehörigen Seitenzahlen in der dritten Glossarspalte.\\
		  number &Die jedem Eintrag zugeordnete Nummerierung. Nummerierungen können sein:\\
		  			&\textbf{page}: jedem Eintrag sind/ist die zugehörige(n) Seitenzahl(en) zugeordnet (default).\\
		  			&\textbf{section}: jedem Eintrag sind/ist die zugehörige(n) Kapitelnummer(n) zugeordnet.\\
		  			&\textbf{none}: es werden den Einträgen keine Nummerierung hinzugefügt.\\
		  toc &Legt fest, ob das Glossar in das Inhaltsverzeichnis aufgenommen werden soll:\\
		  		&\textbf{true}: fügt das Glossar zum Inhaltsverzeichnis hinzu.\\
		  		&\textbf{false}: fügt das Glossar nicht zum Inhaltsverzeichnis hinzu.
\end{tabularx}
\vspace{1em}
\\
Weitere Möglichkeiten, Glossare nach eigenen Vorstellungen zu gestalten, finden sich im
Dokument ''glossary.dvi -- glossary.sty: \DMLLaTeX{} Package to Assist Generating Glossaries''\footnote{Dieses Dokument befindet sich im Mik\TeX"=Verzeichnis: \verb~/miktex/doc/latex/glossary~.}. 
Als durchaus optisch ansprechend und übersichtlich haben sich die Glossar"=Einstellungen erwiesen, wie sie in \cref{subsec:header} angeführt sind.
