%
% Diplomarbeit mit LaTeX
% ===========================================================================
% This is part of the book "Diplomarbeit mit LaTeX".
% Copyright (c) 2002-2008 Tobias Erbsland, Andreas Nitsch
% See the file diplomarbeit_mit_latex.tex for copying conditions.
%

\chapter{Ausstehendes und offene Fragen}
\label{sec:ausstehendes}

\section{Ausstehende Punkte}

Hier werden unsortiert alle Fragen und ausstehende Punkte aufgelistet, welche noch zu beantworten oder zu realisieren sind. Falls du mich bei einem dieser Punkte unterst�tzen kannst, w�rde ich mich �ber deine Mitarbeit freuen.

\begin{itemize}
	\item Ein Kapitel �ber Querverweise.
	\item Eine Erkl�rung zu den Paketen \textbackslash use\-package\{mathptmx\},\\ \textbackslash usepackage[scaled=.92]\{helvet\} und \textbackslash usepackage\{courier\}, sowie �ber die
	Vor- und Nachteile dieser Pakete bei der Darstellung eines PDF-Dokumentes.
	\item Hinweis auf die Koma-Script-Dokumentation \enquote{scrguide} deutlicher unterbringen (genauer Platz innerhalb der MikTeX-Verzeichnisstruktur).
	\item Eine Erkl�rung zur Option \enquote{\!} bei Floats, sowie einen Hinweis darauf, dass Floats ohne Optionen auch sehr gut platziert werden.
	\item Bei den Grundlagen wird bei \ref{Sonderzeichen} gesagt, wie man die reservierten Sonderzeichen benutze, werde sp�ter gesagt - wo? Au�erdem fehlt eine kurze Erkl�rung, wof�r welches Zeichen steht, also Tilde f�r gesch�tztes Leerzeichen etc.
\end{itemize}

\section{Ank�ndigungen}

Des Weiteren m�chte Andreas weitere Kapitel hinzuf�gen (wird wohl im Laufe seiner eigenen Diplomarbeit
in kurzer Zeit geschehen):

\begin{itemize}
	\item Das Erstellen von Pr�sentationen in Form von Beamerfolien mit der ''beamer''-Klasse
	\item Konvertieren von \DMLLaTeX-Dokumenten ins rtf-Format mittels rtf3\DMLLaTeX.
\end{itemize}

\section{Hilfe gesucht}
\label{sec:hilfe-gesucht}

F�r die folgenden Aufgaben suche ich noch Leute, die mich unterst�tzen. Kontaktiere mich per Mail, falls du eine der folgenden Aufgaben gerne l�sen m�chtest. Du erh�ltst dann Zugang zum Versionsverwaltungssystem dieses Dokuments und kannst so deinen Teil zu diesem Dokument beitragen.

\begin{itemize}
	\item Viele Leute haben mir bereits komplette Korrekturen der PDF Version geschickt. Nochmals Vielen Dank an dieser Stelle f�r
	diese Korrekturen. Um die Fehler jedoch alle im Quellcode zu finden und zu korrigieren, fehlt mir einfach die Zeit. Wer mag, kann
	das Dokument verbessern, indem er alle Rechtschreibfehler korrigiert. Dies jedoch direkt im Quellcode, nicht im PDF.
	\item Es fehlen auch noch viele andere Kapitel. Vielleicht m�chtest du ja noch eines schreiben.
\end{itemize}

�ber die URL \href{https://secure.drzoom.ch/svn/dml}{https://secure.drzoom.ch/svn/dml} kannst du mit \enquote{Subversion} jederzeit die Quellen der aktuellste Version des Dokuments holen und deine �nderungen einarbeiten. Um deine �nderungen hochzuladen, ben�tigst du einen Account. Diesen bekommst du bei mir (Tobias). Schreib mir dazu einfach ein Mail, beschreib kurz was du gerne hinzuf�gen m�chtest und wie dein Accountname heissen soll.

Ein guter Subversion Client f�r Windows ist \href{http://tortoisesvn.tigris.org/}{\enquote{TortoiseSVN}}. Weitere Informationen �ber Subversion findest du im \href{http://svnbook.red-bean.com/}{\enquote{Subversion Book}}.

