%
% Diplomarbeit mit LaTeX
% ===========================================================================
% This is part of the book "Diplomarbeit mit LaTeX".
% Copyright (c) 2002-2005 Tobias Erbsland, Andreas Nitsch
% See the file diplomarbeit_mit_latex.tex for copying conditions.
%

\chapter{Einleitung}

\section{Motivation}
\label{sec:motivation}

\begin{quote}
\enquote{Es gibt Alternativen zu WYSIWYG\footnote{What You See Is What You Get} Textverarbeitungen}.
\end{quote}

W�hrend einer Abschlussarbeit steht man meist unter starkem Zeitdruck. Einen gro�en Teil der Zeit, welche du zur Verf�gung hast, brauchst du, um die Dokumentation zu deiner Arbeit zu schreiben. Es lohnt sich, Gedanken dar�ber machen, welches die geeignetste Anwendung f�r ein solch meist umfangreiches Dokument ist. Denn man m�chte schlie�lich vermeiden, seine Zeit mit �rgerlichen Programmabst�rzen, falschen Seitennummerierungen und unerkl�rlichen Effekten, die sich nicht beheben lassen, zu verschwenden.\footnote{Ich beziehe mich in diesen Ausf�hrungen auf Programme wie z.\,B. Microsoft Word. Selbstverst�ndlich gibt es sehr gute WYSIWYG Programme. Es existieren auch sehr gute WYSIWYG-Erweiterungen und -Editoren, welche \DMLLaTeX"=Code direkt grafisch darstellen.}

Meistens beginnen die Probleme ab einer bestimmten Gr��e des Dokuments, aber dann ist es oft zu sp�t, um die Anwendung zu wechseln.

Ich m�chte dir daher einen einfachen Weg aufzeigen, wie du deine Abschlussarbeit oder die Dokumentation dazu mit \DMLLaTeX\footnote{Ausgesprochen wird \DMLLaTeX{} \enquote{laa-tech}, wobei das \emph{X}, der gro�e griechische Buchstabe \emph{Chi}, ein stimmloser uvularer Frikativ ist, also wie in \enquote{ach} oder \enquote{Loch} ausgesprochen werden sollte. Da dieser Laut nach einem \emph{e} jedoch f�r Deutschsprachler ungewohnt ist, wird im deutschsprachigen Raum oft anstelle dessen ein stimmloser palataler Frikativ, also ein Ich-Laut wie in \enquote{Technik} verwendet.} erstellen kannst. Dabei beschreibe ich detailliert den Weg von der Installation einer \DMLLaTeX-Distribution bis zum ersten lauf\/f�higen Dokument (schwerpunktm��ig unter Windows). Weiter beschreibt dieses Dokument h�ufig ben�tigte Formatierungen und Themen, welche im Zusammenhang mit Abschlussarbeiten wie einer Diplomarbeit wichtig sind.

\section{Zu diesem Dokument}

Im vorliegenden Dokument werden Anf�hrungszeichen im Schweizer Stil \enquote{ und } verwendet (im Gegensatz zu den deutschen "`~und~"').

\subsection{Unterst�tzung, Vorschl�ge und Erg�nzungen} %Unterst�tzung hierhin verschoben, weil einiges doppelt anzugeben gewesen w�re 

Ich schreibe dieses Dokument in der Hoffnung, dass es n�tzlich ist. Daher freue ich mich nat�rlich �ber Fehlerberichtigungen und Erg�nzungen, welche in das Konzept dieses Dokuments passen. Falls du mir gerne helfen m�chtest, findest du einige Anregungen und weitere Details im Abschnitt \ref{sec:hilfe-gesucht}.

Bevor du Fehler meldest oder Vorschl�ge machst, solltest du kontrollieren, ob du die neueste Version dieses Dokuments hast, welche du unter

\urlindent[\footnote{Die urspr�ngliche Website http://drzoom.ch/project/dml/ und die Mailingliste existieren nicht mehr.}]{https://github.com/texdoc/diplomarbeit-mit-latex}

findest. Dort kannst du auch Fragen stellen und Vorschl�ge machen.

\subsection{Dank}

Folgende Personen haben mich beim Schreiben dieses Dokumentes unterst�tzt. Ich danke ihnen f�r Korrekturen, Verbesserungen und Kritik. Dadurch ist diese Anleitung wesentlich lesenswerter geworden. 

\begin{itemize}
	\item \enquote{seth}
	\item Christian Faulhammer
	\item Thomas Holenstein
	\item David Kastrup
	\item Markus Kohm
	\item Christian Kuwer
	\item Thomas Ratajczak
	\item Mark Trettin
	\item Uwe Bieling
\end{itemize}

\subsection{Ausstehende und durchgef�hrte �nderungen an diesem Dokument}

Im git-Repository befindet sich eine Datei CHANGELOG.md, welche die �nderungen zwischen den verschiedenen Versionen dieses Dokuments aufzeigt. Daneben findest du eine Liste mit ausstehenden Fragen und �nderungen im Anhang \ref{sec:ausstehendes} des vorliegenden Dokuments.

%
% EOF
%
