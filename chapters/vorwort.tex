%
% Abschlussarbeit mit LaTeX
% ===========================================================================
% This is part of the book "Abschlussarbeit mit LaTeX".
% Copyright (c) 2002-2005 Tobias Erbsland, Andreas Nitsch
% See the file abschlussarbeit_mit_latex.tex for copying conditions.
%

\chapter{Einleitung}

\section{Motivation}
\label{sec:motivation}

\begin{quote}
\enquote{Es gibt Alternativen zu WYSIWYG\footnote{What You See Is What You Get} Textverarbeitungen}.
\end{quote}

Während einer Abschlussarbeit steht man meist unter starkem Zeitdruck. Einen großen Teil der Zeit, welche du zur Verfügung hast, brauchst du, um die Dokumentation zu deiner Arbeit zu schreiben. Es lohnt sich, Gedanken darüber machen, welches die geeignetste Anwendung für ein solch meist umfangreiches Dokument ist. Denn man möchte schließlich vermeiden, seine Zeit mit ärgerlichen Programmabstürzen, falschen Seitennummerierungen und unerklärlichen Effekten, die sich nicht beheben lassen, zu verschwenden.\footnote{Ich beziehe mich in diesen Ausführungen auf Programme wie z.\,B. Microsoft Word. Selbstverständlich gibt es sehr gute WYSIWYG Programme. Es existieren auch sehr gute WYSIWYG-Erweiterungen und -Editoren, welche \DMLLaTeX"=Code direkt grafisch darstellen.}

Meistens beginnen die Probleme ab einer bestimmten Größe des Dokuments, aber dann ist es oft zu spät, um die Anwendung zu wechseln.

Ich möchte dir daher einen einfachen Weg aufzeigen, wie du deine Abschlussarbeit oder die Dokumentation dazu mit \DMLLaTeX\footnote{Ausgesprochen wird \DMLLaTeX{} \enquote{laa-tech}, wobei das \emph{X}, der große griechische Buchstabe \emph{Chi}, ein stimmloser uvularer Frikativ ist, also wie in \enquote{ach} oder \enquote{Loch} ausgesprochen werden sollte. Da dieser Laut nach einem \emph{e} jedoch für Deutschsprachler ungewohnt ist, wird im deutschsprachigen Raum oft anstelle dessen ein stimmloser palataler Frikativ, also ein Ich-Laut wie in \enquote{Technik} verwendet.} erstellen kannst. Dabei beschreibe ich detailliert den Weg von der Installation einer \DMLLaTeX"=Distribution bis zum ersten lauf\/fähigen Dokument (schwerpunktmäßig unter Windows). Weiter beschreibt dieses Dokument häufig benötigte Formatierungen und Themen, welche im Zusammenhang mit Abschlussarbeiten wichtig sind.

\section{Zu diesem Dokument}

Im vorliegenden Dokument werden Anführungszeichen im Schweizer Stil \enquote{ und } verwendet (im Gegensatz zu den deutschen "`~und~"').

\subsection{Unterstützung, Vorschläge und Ergänzungen} %Unterstützung hierhin verschoben, weil einiges doppelt anzugeben gewesen wäre 

Ich schreibe dieses Dokument in der Hoffnung, dass es nützlich ist. Daher freue ich mich natürlich über Fehlerberichtigungen und Ergänzungen, welche in das Konzept dieses Dokuments passen. Falls du mir gerne helfen möchtest, findest du einige Anregungen und weitere Details unter \cref{sec:hilfe-gesucht}.

Bevor du Fehler meldest oder Vorschläge machst, solltest du kontrollieren, ob du die neueste Version dieses Dokuments hast, welche du unter

\urlindent[\footnote{Die ursprüngliche Website http://drzoom.ch/project/dml/ und die Mailingliste existieren nicht mehr.}]{https://github.com/texdoc/diplomarbeit-mit-latex}

findest. Dort kannst du auch Fragen stellen und Vorschläge machen.

\subsection{Dank}

Folgende Personen haben mich beim Schreiben dieses Dokumentes unterstützt. Ich danke ihnen für Korrekturen, Verbesserungen und Kritik. Dadurch ist diese Anleitung wesentlich lesenswerter geworden. 

\begin{itemize}
	\item \enquote{seth}
	\item Christian Faulhammer
	\item Thomas Holenstein
	\item David Kastrup
	\item Markus Kohm
	\item Christian Kuwer
	\item Thomas Ratajczak
	\item Mark Trettin
	\item Uwe Bieling
\end{itemize}

\subsection{Ausstehende und durchgeführte Änderungen an diesem Dokument}

Im git-Repository befindet sich eine Datei CHANGELOG.md, welche die Änderungen zwischen den verschiedenen Versionen dieses Dokuments aufzeigt. Daneben findest du eine Liste mit ausstehenden Fragen und Änderungen im \cref{sec:ausstehendes} des vorliegenden Dokuments.

%
% EOF
%
